\documentclass[letterpaper,11pt]{article}

\usepackage{geometry, pslatex, fancyhdr, graphicx}
\usepackage{amsmath,amsthm,amssymb,scrextend}
\usepackage{multicol}
\usepackage[makeroom]{cancel}
\usepackage{color}
\geometry{ margin = 1.0in }

%%% TODO modify these variables as per your homework %%%
\def\homeworknum{1}
\def\myname{Harshit Jain}
\def\myuserid{hmj5262}
%%%%

\pagestyle{fancy}
\lhead{{\bf CMPSC 461 Spring 2023}}
\chead{{\bf Assignment~\homeworknum}}
\rhead{{\bf \today}}
\let\newproof\proof
\renewenvironment{proof}{\begin{addmargin}[1em]{0em}\begin{newproof}}{\end{newproof}\end{addmargin}\qed}

\newcounter{problemid}
\stepcounter{problemid}
\def\newproblem{\clearpage\newpage{\bf Problem~\arabic{problemid}\stepcounter{problemid}}\hfill\par}

\setlength\parindent{0em}
\setlength\parskip{8pt}
\setlength{\fboxsep}{6pt}


\begin{document}

\framebox[\textwidth]{
	\parbox{0.96\textwidth}{
		\parbox{0.12\textwidth}{\bf Name:}\parbox{0.6\textwidth}{\myname}\\
		\parbox{0.12\textwidth}{\bf User ID:}\parbox{0.6\textwidth}{\myuserid}
	}
}
%% your solutions %%%


% PROBLEM 1
\newproblem 
\begin{enumerate}
    \item (aa)* b (bb)*?
    \item $\epsilon \mid$ a $\mid$ b $\mid$ (a $\mid$ b) (ab $\mid$ ba $\mid$ bb)
    \item (b $\mid$ ab)*a?
    \item (ab*c) $\mid$ (ba*c) $\mid$ (ac*b) $\mid$ (bc*a) $\mid$ (cb*a) $\mid$ (ca*b)
    \item It can not be constructed because regex cannot do storage.
    \item \text{[a-z, A-Z, 0-9]}\{13,18\} @ \text{[a-z, A-Z]} \{4\} ($\backslash$. edu)
    \item .* $\backslash$( $\backslash$( $\backslash$( $\backslash$) .*

\end{enumerate}




% PROBLEM 2
\newproblem 
\begin{enumerate}
    \item Java is both interpreted and compiled. The Java source code is first compiled into bytecode, which is a machine-independent code that runs on the Java Virtual Machine (JVM). The JVM then interprets the bytecode and executes the corresponding machine code.
    
    So, in a sense, Java is compiled into an intermediate form (bytecode) and then interpreted by the JVM. However, the JVM can also use just-in-time (JIT) compilation to dynamically compile the bytecode into machine code for improved performance at runtime.
    
    Therefore, in a broader sense, Java can be considered both compiled and interpreted, depending on the stage of the execution process and the specific implementation of the JVM.

    \item It would not be a DFA if arrows are inverted.
    
    \includegraphics[scale = 0.50]{2.2}
    
    \item DFAs are subsets of NFAs, therefore no conversion needed.
    
    \item It is not possible because DFA has no storage.
    
    \item \includegraphics[scale = 0.50]{2.5}
    
    \item \includegraphics[scale = 0.50]{2.6}




\end{enumerate}




% PROBLEM 3
\newproblem 
\begin{enumerate}
    \item
    $ S \rightarrow aXb \mid bXa $
    
    $ X \rightarrow aX \mid bX \mid \epsilon $

    \item $ S \rightarrow aSbS \mid bSaS \mid \epsilon $
    
    \item Yes, the given grammar is ambiguous because it does not specify a unique parse tree for any string in the language. The unambiguous grammar is:
    
    $<binary-string> \rightarrow <b> \mid <binary-string> <b>$
      
    $<b> \rightarrow 0 \mid 1$

\end{enumerate}




% PROBLEM 4
\newproblem 
\begin{enumerate}
    
    \item $ S \rightarrow S - S $
    
    $ S \rightarrow D - S $

    $ S \rightarrow 6 - S - S $

    $ S \rightarrow 6 - D - S $

    $ S \rightarrow 6 - 3 - D $

    $ S \rightarrow 6 - 3 - 4 $

    \item \includegraphics[scale = 0.25]{4.2}
    
    \item This derivation can continue indefinitely, as there is no rule to reduce the expression to a single number. Therefore, "9 - 1 -" is not a well-formed expression according to this grammar.
    
    \item $ S \rightarrow S - S $
    
    $ S \rightarrow S + S - S $

    $ S \rightarrow S + D - S $

    $ S \rightarrow S + S + 3 - D $

    $ S \rightarrow D + D + 3 - 5 $

    $ S \rightarrow 8 + 6 + 3 - 5 $

    \item Leftmost derivations are those that start at the root of the parse tree and go down the tree towards the leaves, replacing the leftmost non-terminal symbol with its expansion at each step.
    
    Rightmost derivations are those that start at the leaves of the parse tree and go up the tree towards the root, replacing the rightmost non-terminal symbol with its expansion at each step.
    
    There are $4$ leftmost \& rightmost derivations each (total = $8$) for this string because there are $4$ potential starting positions and no branching decisions.

    \includegraphics[scale = 0.60]{4.5}

    All of these trees can be further expanded and can be reached by both right and left most derivations.
\end{enumerate}


 

% PROBLEM 5
\newproblem 
\begin{enumerate}

    \item There are $4$ parse trees possible.
    
    \item No, the number of derivations for the string in Q4.4 is not same as the number of parse trees because everytime left/rightmost have to choose a branch where there is another possible route to be picked. Therefore, the derivation to reach the same tree can vary. 
    
    \item Yes, the grammar is ambiguous because there are multiple parse trees possible and the structure is not uniquely defined and it has associativity.
    
    \item $ S \rightarrow  S + T \mid T $
    
    $ T \rightarrow T - F \mid F $

    $ F \rightarrow D \mid S $

    $D$ is the mathematical digits.

    \item Yes. We can control the precedence of $'+'$ and $'-'$ signs. The grammar in $5.4$ gives $'-'$ sign precedence over $'+'$. Other way is to give precedence to $'+'$.

\end{enumerate}

% PROBLEM 6
\newproblem
\begin{enumerate}

    \item $ S \rightarrow DS \mid ,X \mid \epsilon $
    
    $ X \rightarrow DS $

    $ D \rightarrow 0 \mid 1 \mid 2 \mid 3 \mid 4 \mid 5 \mid 6 \mid 7 \mid 8 \mid 9 $
    
    It is unambiguous because the grammar is right recursive and right associated.

    \item Yes, this grammar is ambiguous because there are multiple parse trees possible. Here's is an example: \includegraphics[scale = 0.75]{6.2} 

    The unambiguous grammar is:

    $ S \rightarrow  aS \mid T $
    
    $ T \rightarrow Tb \mid F $

    $ F \rightarrow b \mid \epsilon $

    \item $ E \rightarrow  E + T \mid T $
    
    $ T \rightarrow T - F \mid F $

    $ F \rightarrow \neg F \mid D $

    $ D \rightarrow 0 \mid 1 \mid 2 \mid 3 \mid 4 \mid 5 \mid 6 \mid 7 \mid 8 \mid 9 $

    \item Yes, context-free grammars can represent all regular languages.
    
    No, because regular languages does not translate all grammar.
    
    No, Context-free grammars cannot represent all possible languages (i.e. non-regular).
\end{enumerate}

\end{document}