\documentclass[letterpaper,11pt]{article}

\usepackage{geometry, pslatex, fancyhdr, graphicx}
\usepackage{amsmath,amsthm,amssymb,scrextend}
\usepackage[makeroom]{cancel}
\usepackage{color}
\geometry{ margin = 1.0in }

%%% TODO modify these variables as per your homework %%%
\def\homeworknum{1}
\def\myname{Harshit Jain}
\def\myuserid{hmj5262}
%%%%

\pagestyle{fancy}
\lhead{{\bf CMPSC 461 Spring 2023}}
\chead{{\bf Assignment~\homeworknum}}
\rhead{{\bf \today}}
\let\newproof\proof
\renewenvironment{proof}{\begin{addmargin}[1em]{0em}\begin{newproof}}{\end{newproof}\end{addmargin}\qed}

\newcounter{problemid}
\stepcounter{problemid}
\def\newproblem{\clearpage\newpage{\bf Problem~\arabic{problemid}\stepcounter{problemid}}\hfill\par}

\setlength\parindent{0em}
\setlength\parskip{8pt}
\setlength{\fboxsep}{6pt}


\begin{document}

\framebox[\textwidth]{
	\parbox{0.96\textwidth}{
		\parbox{0.12\textwidth}{\bf Name:}\parbox{0.6\textwidth}{\myname}\\
		\parbox{0.12\textwidth}{\bf User ID:}\parbox{0.6\textwidth}{\myuserid}
	}
}
%% your solutions %%%

% PROBLEM 1
\newproblem 
\begin{enumerate}
    \item (aa)* (b (aa)*)* OR ((aa)* $\mid$ b(bb)*)*
    \item ((a$\mid$b)* (b)) $\mid$ (((a$\mid$b)(b))* (a)) 
    \item (a?b)*
    \item 
    \item It can not be constructed.
    \item \text{[a-z, A-Z, 0-9]}\{13,18\} @ \text{[a-z, A-Z]} \{4\} (.edu)
    \item 
\end{enumerate}




% PROBLEM 2
\newproblem 
\begin{enumerate}
    \item Java is both interpreted and compiled. The Java source code is first compiled into bytecode, which is a machine-independent code that runs on the Java Virtual Machine (JVM). The JVM then interprets the bytecode and executes the corresponding machine code.
    
    So, in a sense, Java is compiled into an intermediate form (bytecode) and then interpreted by the JVM. However, the JVM can also use just-in-time (JIT) compilation to dynamically compile the bytecode into machine code for improved performance at runtime.
    
    Therefore, in a broader sense, Java can be considered both compiled and interpreted, depending on the stage of the execution process and the specific implementation of the JVM.
\end{enumerate}



% PROBLEM 3
\newproblem 





% PROBLEM 4
\newproblem 






% PROBLEM 5
\newproblem 





% PROBLEM 6
\newproblem 


\end{document}